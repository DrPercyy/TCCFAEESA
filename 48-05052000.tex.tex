\documentclass[addpoints,answers]{exam}%
\usepackage[T1]{fontenc}%
\usepackage[utf8]{inputenc}%
\usepackage{lmodern}%
\usepackage{textcomp}%
\usepackage{lastpage}%
\usepackage{graphicx}%
\usepackage{ragged2e}%
%
%
%
\begin{document}%
\normalsize%
\ID Prova: {0}%


\begin{figure}[htbp]%
\centering%
\begin{tabular}{p{5.5in}}%
\textbf{Aluno: Julio Cesar Lima {-} Data : 05/05/2000}\\%
\hline%
\textbf{48{-}05052000 {-} Prova integrada}\\%
\end{tabular}%
\begin{minipage}{1\textwidth}%
\includegraphics[width=0.2\textwidth]{qrcode.png}%
\label{fig:qrcode}%
\end{minipage}%
\end{figure}

%
\centering%
\section*{Questionário}%
\label{sec:Questionrio}%

%
\begin{questions}%
\question{Qual das seguintes opções é um exemplo de armazenamento de dados em nuvem?}%
\begin{choices}%
\choice{Disco rígido externo.}%
\choice{Pen drive USB.}%
\choice{Servidor local.}%
\choice{Google Drive.}%
\choice{DVD virgem.}%
\end{choices}%
\question{O que é a memória RAM (Random Access Memory) em um computador?}%
\begin{choices}%
\choice{Uma unidade de armazenamento permanente para dados.}%
\choice{Um dispositivo de entrada para informações digitais.}%
\choice{Uma parte do processador que executa cálculos complexos.}%
\choice{Uma memória temporária que armazena dados enquanto o computador está ligado.}%
\choice{Uma unidade de processamento gráfico.}%
\end{choices}%
\question{O que representa a tabela verdade em lógica?}%
\begin{choices}%
\choice{Uma lista de proposições que não têm valor lógico definido.}%
\choice{Um método para criar argumentos inválidos.}%
\choice{Uma representação visual das operações lógicas entre proposições.}%
\choice{Uma tabela de valores lógicos que mostra todas as combinações possíveis de valores para proposições.}%
\choice{Uma lista de falácias lógicas comuns.}%
\end{choices}%
\question{O que é um arquivo no contexto de computação?}%
\begin{choices}%
\choice{Uma unidade de armazenamento de dados dentro do processador.}%
\choice{Um dispositivo de entrada de dados, como um teclado.}%
\choice{Um conjunto de instruções para um programa de computador.}%
\choice{Uma unidade organizacional em um sistema operacional.}%
\choice{Uma coleção de dados armazenados em um meio de armazenamento, como um disco rígido.}%
\end{choices}%
\question{Qual é a equivalência tautológica que representa a Lei da Comutatividade na álgebra booleana?}%
\begin{choices}%
\choice{p ∨ q ≡ q ∨ p}%
\choice{p ∨ (q ∨ r) ≡ (p ∨ q) ∨ r}%
\choice{p ∨ p ≡ p}%
\choice{p ∧ (q ∨ r) ≡ (p ∧ q) ∨ r}%
\choice{¬(p ∧ q) ≡ ¬p ∧ ¬q}%
\end{choices}%
\question{O que é um disco rígido (HD) em um computador?}%
\begin{choices}%
\choice{Uma unidade de processamento central.}%
\choice{Um dispositivo de entrada de dados.}%
\choice{Uma unidade de armazenamento de dados não volátil que utiliza discos magnéticos para armazenar informações.}%
\choice{Um dispositivo de saída de dados.}%
\choice{Uma memória RAM.}%
\end{choices}%
\question{O que é a programação orientada a objetos?}%
\begin{choices}%
\choice{Uma técnica para criar programas sem a necessidade de código.}%
\choice{Um método de programação que não usa estruturas de controle.}%
\choice{Um paradigma de programação que se baseia na criação de objetos que podem interagir entre si.}%
\choice{Uma forma de programação que não requer linguagens de programação.}%
\choice{Um tipo de programação usada apenas em jogos de computador.}%
\end{choices}%
\question{Quantas linhas uma tabela verdade deve ter para representar todas as combinações de valores lógicos para duas proposições?}%
\begin{choices}%
\choice{1}%
\choice{2}%
\choice{3}%
\choice{4}%
\choice{5}%
\end{choices}%
\question{O que representa o valor "Falso" (False) em uma tabela verdade?}%
\begin{choices}%
\choice{A proposição é verdadeira.}%
\choice{A proposição é falsa.}%
\choice{A proposição é indefinida.}%
\choice{A proposição é uma tautologia.}%
\choice{A proposição é contraditória.}%
\end{choices}%
\question{Qual é a equivalência tautológica que representa a Lei da Dupla Negação na lógica proposicional?}%
\begin{choices}%
\choice{p ≡ ¬(¬p)}%
\choice{p ≡ ¬p}%
\choice{¬p ≡ ¬(¬p)}%
\choice{p ≡ (¬p)}%
\choice{¬p ≡ (¬¬p)}%
\end{choices}%
\end{questions}%
\section*{Gabarito}%
\label{sec:Gabarito}%

%
\begin{tabular}{|c|c|}%
\hline%
Questão&Resposta\\%
\hline%
1&\\%
\hline%
2&\\%
\hline%
3&\\%
\hline%
4&\\%
\hline%
5&\\%
\hline%
\end{tabular}%
\begin{tabular}{|c|c|}%
\hline%
Questão&Resposta\\%
\hline%
6&\\%
\hline%
7&\\%
\hline%
8&\\%
\hline%
9&\\%
\hline%
10&\\%
\hline%
\end{tabular}%
\end{document}%
\end{document}